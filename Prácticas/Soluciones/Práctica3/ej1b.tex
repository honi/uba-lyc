$f_2(x, y) = \begin{cases}
    1 & \text{si } \Phi_x^{(1)}(y) = 0 \\
    0 & \text{si no}
\end{cases}$

Suponemos que $f_2$ es computable, y por lo tanto existe un programa $P_2$ que la computa.

Definimos $g(x) = f_2(x, x) = \begin{cases}
    1 & \text{si } \Phi_x^{(1)}(x) = 0 \\
    0 & \text{si no}
\end{cases}$

Sea $Q$ un programa que computa $g$:

\begin{lstlisting}
    $X_2 \gets X_1$ (macro computable)
    $P_2$
\end{lstlisting}

Como asumimos que $f_2$ es computable, $g$ también lo es pues $f_2$ computable $\implies g$ computable.

Definimos ahora un programa $Q'$ tal que $\Psi_{Q'}(x) = \begin{cases}
    0 & \text{si } \Phi_x^{(1)}(x) > 0 \lor \Phi_x^{(1)}(x) \uparrow \\
    \uparrow & \text{si no}
\end{cases}$

\begin{lstlisting}
    $Q$
[R] IF $Y$ $\neq$ 0 GOTO R
\end{lstlisting}

Veamos que $\forall x: \Phi_{\#(Q')}^{(1)}(x) \downarrow \iff \Psi_{Q'}(x) \downarrow \iff \Psi_Q(x) = 0 \iff g(x) = 0 \iff \Phi_x^{(1)}(x) > 0 \lor \Phi_x^{(1)}(x) \uparrow$

Sea $e = \#(Q')$ y tomando $x = e$ vemos que: $\Phi_e^{(1)}(e) \downarrow \iff \Phi_e^{(1)}(e) > 0 \lor \Phi_e^{(1)}(e) \uparrow$

Analizamos cada caso:

\begin{itemize}
    \item $\Phi_e^{(1)}(e) \downarrow \iff \Phi_e^{(1)}(e) > 0$ Absurdo pues $\Phi_e^{(1)}(e) \downarrow \iff \Phi_e^{(1)}(e) = 0$ (por definición)
    \item $\Phi_e^{(1)}(e) \downarrow \iff \Phi_e^{(1)}(e) \uparrow$ Absurdo
\end{itemize}

Llegamos al absurdo por suponer que $f_2$ era computable. Por lo tanto, $f_2$ no es computable.
