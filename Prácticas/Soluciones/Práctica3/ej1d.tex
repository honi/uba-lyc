$f_4(x) = \begin{cases}
    1 & \text{si } \Phi_x^{(1)}(x) \downarrow \land \; \Phi_x^{(1)}(x) \neq x \\
    0 & \text{si no}
\end{cases}$

Suponemos que $f_4$ es computable, y por lo tanto existe un programa $P_4$ que la computa.

Definimos ahora un programa $Q$ tal que $\Psi_{Q}(x) = \begin{cases}
    0 & \text{si } \Phi_x^{(1)}(x) \uparrow \lor \; \Phi_x^{(1)}(x) = x \\
    \uparrow & \text{si no}
\end{cases}$

\begin{lstlisting}
    $P_4$
[R] IF $Y$ $\neq$ 0 GOTO R
\end{lstlisting}

Veamos que $\forall x: \Phi_{\#(Q)}^{(1)}(x) \downarrow \iff \Psi_{P_4}(x) = 0 \iff f_4(x) = 0 \iff \Phi_x^{(1)}(x) \uparrow \lor \; \Phi_x^{(1)}(x) = x$

Sea $e = \#(Q)$ y tomando $x = e$ vemos que: $\Phi_e^{(1)}(e) \downarrow \iff \Phi_e^{(1)}(e) \uparrow \lor \; \Phi_e^{(1)}(e) = e$

Analizamos cada caso:

\begin{itemize}
    \item $\Phi_e^{(1)}(e) \downarrow \iff \Phi_e^{(1)}(e) = e$ Absurdo pues $\Phi_e^{(1)}(e) \downarrow \iff \Phi_e^{(1)}(e) = 0 \neq e$ pues $Q$ no es el programa vacío
    \item $\Phi_e^{(1)}(e) \downarrow \iff \Phi_e^{(1)}(e) \uparrow$ Absurdo
\end{itemize}

Llegamos al absurdo por suponer que $f_4$ era computable. Por lo tanto, $f_4$ no es computable.
