\section*{Ejercicio 3}

\subsection*{a)}

Para la ida $(\implies)$ hacemos una demostración por inducción estructural. Primero probamos que todas las funciones iniciales (que están en $\C_c$) cumplen la propiedad.

\begin{itemize}
    \item Función nula

    $f(x) = n(x) = 0$. La función nula cae en el caso $f(x) = k$ donde $k = 0$.

    \item Función sucesor

    $f(x) = s(x) = x + 1$. La función sucesor cae en el caso $f(x) = x + k$ donde $k = 1$.

    \item Función proyector

    $f(x_1, \dots, x_n) = u^n_i(x_1, \dots, x_n) = x_i$. La función proyector cae en el caso $f(x_1, \dots, x_n) = x_i + k$ donde $k = 0$.
\end{itemize}

Paso inductivo. Supongamos que existen $f, g_1, \dots, g_m \in \C_c$ que cumplen con la propiedad. Como la única operación en $\C_c$ es la composición, esa es la única forma de generar nuevas funciones. Probemos entonces que la función generada por composición $h(x_1, \dots, x_n) = f(g_1(x_1, \dots, x_n), \dots, g_m(x_1, \dots, x_n)) \in \C_c$ también cumple con la propiedad.

Miremos todos los posibles casos para $f$ que ya sabemos que cumple con la propiedad por hipótesis inductiva.

\begin{itemize}
    \item $f(g_1(x_1, \dots, x_n), \dots, g_m(x_1, \dots, x_n)) = k \implies h(x_1, \dots, x_n) = k$

    \item $f(g_1(x_1, \dots, x_n), \dots, g_m(x_1, \dots, x_n)) = g_j(x_1, \dots, x_n) + k_0$ para algún $j \mid 1 \leq j \leq m$.

    Por hipótesis inductiva, $g_j \in \C_c$ cumple con la propiedad. Por lo tanto $g_j(x_1, \dots, x_n) = k_j$ o bien $g_j(x_1, \dots, x_n) = x_i + k_j$ para algún $i \mid 1 \leq i \leq n$. Observar que necesitamos identificar $k_j$ con un subíndice ya que cada $g_j$ puede tener constantes arbitrarias. Además, necesitamos usar un subíndice distinto para el $x_i$ ya que sino estaríamos limitando qué parámetro de entrada puede usar cada $g_j$.

    \begin{itemize}
        \item $g_j(x_1, \dots, x_n) = k_j \implies h(x_1, \dots, x_n) = k_j + k_0 = k$
        \item $g_j(x_1, \dots, x_n) = x_i + k_j \implies h(x_1, \dots, x_n) = x_i + k_j + k_0 = x_i + k$
    \end{itemize}
\end{itemize}

Probamos entonces que cualquier función $h(x_1, \dots, x_n) = f(g_1(x_1, \dots, x_n), \dots, g_m(x_1, \dots, x_n)) \in \C_c$ cumple con la propiedad. Es decir, vale que si $h \in \C_c \implies h(x_1, \dots, x_n) = k$ o bien $h(x_1, \dots, x_n) = x_i + k$.

Para la vuelta $(\impliedby)$ mostramos que podemos construir cualquier $f(x_1, \dots, x_n) = k$ o $f(x_1, \dots, x_n) = x_i + k$ a partir de composición de las funciones iniciales, y por lo tanto $f \in \C_c$.

\begin{itemize}
    \item $f(x_1, \dots, x_n) = k = s^k(n(x_i))$ para cualquier $i$
    \item $f(x_1, \dots, x_n) = x_i + k = s^k(u^n_i(x_1, \dots, x_n))$
\end{itemize}

\subsection*{b)}

En el ejercicio 2 vimos que la función $\text{suma}(x, y) = x + y$ es P.R. pero $\text{suma} \notin \C_c$ pues no cumple con la propiedad.
