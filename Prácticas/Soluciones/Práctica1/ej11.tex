\section*{Ejercicio 11}

Definimos una función $H$ que va a realizar la recursión mutua codificándola en un par.

$H(\x, t) = \tupla{h_1(\x, t)}{h_2(\x, t)}$

Planteamos el esquema de recursión primitiva.

$H(\x, 0) = \tupla{h_1(\x, 0)}{h_2(\x, 0)} = \tupla{f_1(\x)}{f_2(\x)}$

$\begin{aligned}
H(\x, t + 1)
&= \tupla{h_1(\x, t + 1)}{h_2(\x, t + 1)} \\
& = \tupla{g_1(h_1(\x, t), h_2(\x, t), \x, t)}{g_2(h_2(\x, t), h_1(\x, t), \x, t)} \\
& = \tupla{g_1(l(H(\x, t)), r(H(\x, t)), \x, t)}{g_2(r(H(\x, t)), l(H(\x, t)), \x, t)}
\end{aligned}$

$H$ sigue el esquema de recursión primitiva y es composiciones de funciones p.r. y de $f_1, f_2, g_1, g_2 \in \C \implies H \in \C$.

Como $H \in \C$ está codificando a $h_1$ y $h_2$ de la siguiente forma:

$h_1(\x, t) = l(H(\x, t))$

$h_2(\x, t) = r(H(\x, t))$

$\implies h_1, h_2 \in \C$.
