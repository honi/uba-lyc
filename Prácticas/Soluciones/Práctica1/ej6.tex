\section*{Ejercicio 6}

Una clase de funciones $\C$ es PRC si contiene las funciones iniciales y está cerrada por composición y recursión primitiva.

Podemos afirmar que una función cualquiera va a estar en \textbf{toda} clase PRC si podemos demostrar que pertenece a la clase PR. La clase PR es la clase PRC más ``chica'', son todas las funciones que se pueden construir a partir de las funciones iniciales mediante composición y/o recursión primitiva, y por lo tanto van a pertenecer a cualquier clase PRC.

No obstante, una clase $\C$ puede ser PRC y a su vez contener funciones que no puedan ser construidas a partir de las iniciales. Esto sucede cuando la clase incluye explícitamente alguna función adicional que no pertenece a la clase PR. En esencia, para generar nuevas funciones dentro de esta clase $\C$, además de tener las 3 funciones iniciales, tendríamos esta función adicional.

\subsection*{a)}

$\text{par}(x) = \begin{cases}
    1 & \text{si $x$ es par} \\
    0 & \text{si no}
\end{cases}$

$\text{par}(0) = 1$

$\text{par}(x + 1) = g(\text{par}(x), x)$ donde $g(x_1, x_2) = \neg u^2_1(x_1, x_2) \implies \text{par}(x + 1) = \neg \text{par}(x)$

Definimos $\text{impar}(x) = \neg \text{par}(x)$ para usar en los siguientes items.

\subsection*{b)}

$f(x) = \text{div2}(x) = \lfloor x / 2 \rfloor$

$\text{div2}(0) = 0$

$\text{div2}(x + 1) = \begin{cases}
    \text{div2}(x) & \text{si } \text{par}(x) \\
    s(\text{div2}(x)) & \text{si no}
\end{cases} = \text{div2}(x) \cdot \text{par}(x) + s(\text{div2}(x)) \cdot \text{impar}(x)$

\subsection*{c)}

$h(x_1, \dots, x_n, t) = \begin{cases}
    f(x_1, \dots, x_n) & \text{si } t = 0 \\
    g_1(x_1, \dots, x_n, k, h(x_1, \dots, x_n, t - 1)) & \text{si } t = 2 \cdot k + 1 \\
    g_2(x_1, \dots, x_n, k, h(x_1, \dots, x_n, t - 1)) & \text{si } t = 2 \cdot k + 2
\end{cases}$

Planteamos el caso base del esquema de recursión primitiva para $h$. Notar que si $t = 0$ siempre entramos en el primer caso de $h$ pues no existe $k \in \N$ para satisfacer los otros 2 casos cuando $t = 0$.

$h(x_1, \dots, x_n, 0) = f(x_1, \dots, x_n)$

Para el caso $t > 0$ primero reescribimos la función por casos colocando $t + 1$ en donde antes teníamos $t$ para poder ajustarnos al esquema de recursión primitiva.

$
\begin{aligned}
h(x_1, \dots, x_n, t + 1)
& = \begin{cases}
    g_1(x_1, \dots, x_n, k, h(x_1, \dots, x_n, t + 1 - 1)) & \text{si } t + 1 = 2 \cdot k + 1 \\
    g_2(x_1, \dots, x_n, k, h(x_1, \dots, x_n, t + 1 - 1)) & \text{si } t + 1 = 2 \cdot k + 2
\end{cases} \\
& = \begin{cases}
    g_1(x_1, \dots, x_n, k, h(x_1, \dots, x_n, t)) & \text{si } t = 2 \cdot k \\
    g_2(x_1, \dots, x_n, k, h(x_1, \dots, x_n, t)) & \text{si } t = 2 \cdot k + 1
\end{cases}
\end{aligned}
$

Ahora necesitamos despejar $k$ en función de $t$.

$t = 2 \cdot k \implies k = \lfloor t / 2 \rfloor = \text{div2}(t)$

$t = 2 \cdot k + 1 \implies k = \lfloor (t - 1) / 2 \rfloor = \text{div2}(t - 1) = \text{div2}(t)$ pues $t$ es impar y div2 redondea hacia abajo

Reescribimos $h$ eliminando por completo la $k$.

$h(x_1, \dots, x_n, t + 1) = \begin{cases}
    g_1(x_1, \dots, x_n, \text{div2}(t), h(x_1, \dots, x_n, t)) & \text{si } \text{par}(t) \\
    g_2(x_1, \dots, x_n, \text{div2}(t), h(x_1, \dots, x_n, t)) & \text{si } \text{impar}(t)
\end{cases}$

También podemos escribir $h$ como suma de cada caso por su predicado.

$h(x_1, \dots, x_n, t + 1)
= g_1(x_1, \dots, x_n, \text{div2}(t), h(x_1, \dots, x_n, t)) \cdot \text{par}(t)
+ g_2(x_1, \dots, x_n, \text{div2}(t), h(x_1, \dots, x_n, t)) \cdot \text{impar}(t)$

\textbf{Conclusión}

\begin{itemize}
    \item $h$ sigue el esquema de recursión primitiva.
    \item $h$ compone funciones que están en $\C$ (puntualmente $f$, $g_1$ y $g_2$).
    \item $h$ compone funciones que están en toda clase PRC (puntualmente div2, par, impar).
    \item $\C$ es una clase PRC por el enunciado, entonces contiene a las funciones div2, par, impar.
\end{itemize}

Por lo tanto cualquier $h$ que cumpla con este esquema pertenece a $\C$.
