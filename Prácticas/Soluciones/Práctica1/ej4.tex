\section*{Ejercicio 4}

Cualquier clase PRC contiene las funciones iniciales y está cerrada por recursión primitiva y composición. Para mostrar que los predicados están en cualquier clase PRC, es suficiente con mostrar que se pueden construir a partir de las funciones iniciales utilizando recursión primitiva y/o composición.

Para simplificar la escritura vamos a utilizar la función $\alpha(x)$ la cual es PR a partir de las iniciales y por lo tanto pertenece a cualquier clase PRC.

$\alpha(x) = \begin{cases}
    1 & \text{si } x = 0 \\
    0 & \text{si no}
\end{cases}$

Además, podemos definir el predicado $\neg(x) = \alpha(x)$ que niega otro predicado.

\begin{itemize}
    \item[$\leq$)] $p(x, y) = \begin{cases}
        1 & \text{si } x \leq y \\
        0 & \text{si no}
    \end{cases} = \alpha(x \dotdiv y)$

    \item[$\geq$)] $p(x, y) = \begin{cases}
        1 & \text{si } x \geq y \\
        0 & \text{si no}
    \end{cases} = \alpha(y \dotdiv x)$

    \item[$=$)] $p(x, y) = \begin{cases}
        1 & \text{si } x = y \\
        0 & \text{si no}
    \end{cases} = (x \leq y) \cdot (x \geq y)$

    \item[$\neq$)] $p(x, y) = \begin{cases}
        1 & \text{si } x \neq y \\
        0 & \text{si no}
    \end{cases} = \neg(x = y)$

    \item[$<$)] $p(x, y) = \begin{cases}
        1 & \text{si } x < y \\
        0 & \text{si no}
    \end{cases} = \neg(x \geq y)$

    \item[$>$)] $p(x, y) = \begin{cases}
        1 & \text{si } x > y \\
        0 & \text{si no}
    \end{cases} = \neg(x \leq y)$
\end{itemize}
