\section*{Ejercicio 7}

Usamos la notación $\x = x_1, \dots, x_n$ para simplificar la escritura. Ya vimos en la teórica que los operadores acotados están necesariamente acotados por arriba. La idea de este ejercicio es mostrar que también podemos acotarlos por abajo.

\begin{itemize}
    \item $\text{cantidad}_p(\x, y, z) = \sum_{t=0}^z p(\x, t) \cdot (t \geq y) = \sum_{t=y}^z p(\x, t)$

    \item $\text{todos}_p(\x, y, z) = (\forall t)_{\leq z} \; (p(\x, t) \cdot (t \geq y) + (t < y)) = (\forall t)_{y \leq t \leq z} \; p(\x, t)$

    Otra forma: $\text{todos}_p(\x, y, z) = \text{cantidad}_p(\x, y, z) = z - y$

    \item $\text{alguno}_p(\x, y, z) = (\exists t)_{\leq z} \; (p(\x, t) \cdot (t \geq y) + (t < y)) = (\exists t)_{y \leq t \leq z} \; p(\x, t)$

    Otra forma: $\text{alguno}_p(\x, y, z) = \text{cantidad}_p(\x, y, z) > 0$

    \item Recordemos la definición de la minimización acotada: $\text{min}_{t \leq y} \; p(\x, t) = \sum_{u=0}^y \prod_{t=0}^u \alpha(p(\x, t))$

    $\text{mínimo}_p(\x, y, z) = \text{min}_{t \leq z} \; (p(\x, t) \cdot (t \geq y)) = \text{min}_{y \leq t \leq z} \; p(\x, t)$

    \item Podemos definir el máximo utilizando el mínimo, agregando una condición adicional al predicado: que no exista ningún otro $t' > t$ para el cual también valga el predicado $p$. En esencia, buscamos primero el mínimo real, y si existe un $t$ más grande para el cual también vale el predicado, tenemos que seguir buscando el próximo mínimo a partir del encontrado, y así hasta eventualmente llegar al máximo.

    $\text{máximo}_p(\x, y, z) = \text{mínimo}_q(\x, y, z)$ donde $q(\x, t) = p(\x, t) \cdot \neg (\exists t')_{t < t' \leq z} \; p(\x, t')$

    \item Sea $m = \text{mínimo}_p(\x, y, z)$, $M = \text{máximo}_p(\x, y, z)$

    $\text{único}_p(\x, y, z) = (m = M) \cdot m + (m \neq M) \cdot (z + 1)$
\end{itemize}
